\documentclass{article}
\usepackage{graphicx} % Required for inserting images
\usepackage{hyperref}
\hypersetup{
    colorlinks=true,
    linkcolor=blue,
    filecolor=magenta,      
    urlcolor=cyan,
    pdftitle={Overleaf Example},
    pdfpagemode=FullScreen,
    }

\urlstyle{same}
\title{838L Proposal Draft + Check-in}
\author{Yusuf Bham \\ Arjun Vedantham }
\date{February 2024}

\begin{document}

\maketitle

\section{Introduction}
We are examining ways to construct a domain specific language to perform signal processing tasks. This would be similar to toolkits like GNURadio, but instead of representing the language as an graphical flowgraph, we would use a more traditional (likely functional) programming paradigm and construct a compiler for the language, rather than just generating C++/Python. Ideally, we would be able to compile this language to an HDL (or maybe CirctIR) so that it could be used with an FPGA to create hardware accelerators for specific tasks (e.g. running a low pass filter over an FM radio signal received from a radio receiver module). 

\section{Additional Resources}
\begin{itemize}
    \item \href{https://www.microsoft.com/en-us/research/wp-content/uploads/2016/02/ASPLOS-camera.pdf}{Ziria} - a DSL specifically for implementing physical layer networking protocols (2015)
    \item \href{https://link.springer.com/chapter/10.1007/978-3-319-51676-9_12}{A Domain-Specific Language for Software-Defined Radio} - builds off of the work of the original Ziria paper by defining semantics in greater detail (2016)
    \item \href{https://www.cs.unb.ca/tech-reports/honours-theses/Matthew.Gordon-4997.pdf}{$\mu$: A Functional Programming Language for Digital Signal Processing}
     Compiles to C++, CPU-bound - also primarily intended for audio synthesis/engineering (2003)
\end{itemize}
\end{document}
