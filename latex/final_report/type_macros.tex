% Largely sourced from https://arxiv.org/abs/1306.6032

\def\CompactJudgments{0}
\newcommand{\entails}{\mathrel{\ifnum\CompactJudgments=1%
    \vdash%
  \else%
     \vdash\,%
  \fi}}
\newcommand{\ctxoutsym}{\ifnum\CompactJudgments=1%
    \dashv%
  \else%
     \,\dashv%
  \fi}
\newcommand{\ctxout}[1]{\mathrel{\ctxoutsym}{#1}}

\definecolor{dHilite}{rgb}{0.9, 0.9, 0.6}
\definecolor{dRed}{rgb}{0.65, 0.0, 0.0}
\definecolor{dGreen}{rgb}{0.0, 0.65, 0.0}
\definecolor{dDkGreen}{rgb}{0.0, 0.35, 0.0}
\definecolor{dBlue}{rgb}{0.0, 0.0, 0.65}
\definecolor{dPurple}{rgb}{0.65, 0.0, 0.65}
\definecolor{dDigPurple}{rgb}{0.5, 0.0, 0.5}
\definecolor{dAlsoPurple}{RGB}{137, 101, 255}
\definecolor{dFaint}{rgb}{0.7, 0.7, 0.7}
\definecolor{dGray}{rgb}{0.5, 0.5, 0.5}
\definecolor{dDark}{rgb}{0.2, 0.2, 0.2}
\definecolor{dAlmostBlack}{rgb}{0.1, 0.1, 0.1}

\newcommand{\rulename}[1]{\text{\normalfont\textsf{#1}}}


\newcommand{\emptysig}{\cdot}
\newcommand{\emptyctx}{\cdot}

\newcommand{\natzero}{\mathsf{zero}}
\newcommand{\xnatsucc}{\mathsf{succ}}
\newcommand{\natsucc}[1]{\xnatsucc\texttt{(}{#1}\texttt{)}}


\newcommand{\instantiate}[1]{{#1}\texttt{[-]}}
\newcommand{\quantify}[1]{\Lambda{#1}.\,}


\newcommand{\exvar}[1]{\widehat{#1}}
\newcommand{\exalpha}{\exvar{\alpha}}
\newcommand{\exbeta}{\exvar{\beta}}

\newcommand{\alln}[1]{\forall{#1}{:}\kindnat.\,}

\newcommand{\Match}[2]{{#1} \Rightarrow {#2}}
\newcommand{\matchor}{\ensuremath{\normalfont\,\texttt{|}\hspace{-5.35pt}\texttt{|}\,}}
\newcommand{\ind}[3]{\mathsf{ind}\texttt{(}\Match{\natzero}{#1}%
                     \texttt{,\;}%
                     \Match{\natsucc{#2}}{#3}
                     \texttt{)}}

\newcommand{\exunk}[2]{{#1} : {#2}}
\newcommand{\exsol}[3]{{#1} : {#2} \texttt{=} {#3}}
\newcommand{\exsolnokind}[2]{{#1} \texttt{=} {#2}}
\newcommand{\exsolwild}[2]{({#1} : {#2}\dots)}
\newcommand{\rexvar}[2]{\exsolnokind{#1}{#2}}

\newcommand{\tyname}[1]{\textsf{\normalfont #1}}
%
\newcommand{\unitexp}{\text{\normalfont \tt()}}
\newcommand{\unitty}{\tyname{\normalfont \tt()}}

\newcommand{\numty}{\tyname{\ensuremath{\vartheta}}}
\newcommand{\usn}{\texttt{u}}
\newcommand{\sn}{\texttt{i}}
\newcommand{\boolty}{\tyname{Bool}}
\newcommand{\isnum}[1]{\mathsf{num}(#1)}
\newcommand{\vecty}[2]{\operatorname{\mathsf{Vec}}(#1, #2)}
\newcommand{\evecty}{\operatorname{\mathsf{Vec}}}
\newcommand{\faty}[2]{\forall #1.\,#2}

\newcommand{\trientails}{\mathrel{{\rhd}\,}}
\newcommand{\trictxoutsym}{{\lhd}}
\newcommand{\trictxout}[1]{\mathrel{\trictxoutsym}{#1}}

\newcommand{\subtypingycolor}[1]{\textcolor{dDigPurple}{#1}}
%
\newcommand{\subtype}{\mathrel{\normalfont\texttt{\subtypingycolor{<:}}}}  %
\newcommand{\declsubtype}{\mathrel{\leq}}

\newcommand{\declsubjudg}[3]{\ensuremath{{#1} \entails {#2} \declsubtype {#3}}}
\newcommand{\typejudg}[3]{\ensuremath{{#1} \entails {#2} : {#3}}}
\newcommand{\subjudg}[4]{\ensuremath{{#1} \entails {#2} \subtype {#3} \ctxout{#4}}}

\newcommand{\subrulename}[1]{\ensuremath{{\subtype}\rulename{#1}}\xspace}
\newcommand{\Infer}[3]{\inferrule*[right={\text{\strut#1}}]{{}#2\mathstrut}{{}#3\mathstrut}}

\newcommand{\textgraybox}[1]{\boxed{#1}}
\newcommand{\judgboxfontsize}[1]{%
    % \ifnum\OPTIONConf=1%
        \mathsz{14pt}{#1}%
    %\else%
        % \mathsz{14pt}{#1}%
    }
    %\fi
\newcommand{\judgbox}[2]{%
      {\raggedright \textgraybox{\ensuremath{\judgboxfontsize{#1}}}\!%
        \fontsz{12pt}{\begin{tabular}[c]{l} #2 \end{tabular}} %
}}

\newdimen\zzfontsz
\newcommand{\fontsz}[2]{\zzfontsz=#1%
{\fontsize{\zzfontsz}{1.2\zzfontsz}\selectfont{#2}}}

\newcommand{\mathsz}[2]{\text{\fontsz{#1}{$#2$}}}

\DeclareMathOperator*{\bigentails}{\,\,\entails}
\newcommand{\judgetp}[2]{{#1} \entails {#2}}
\newcommand{\judgetps}[3]{{#1} \bigentails_{i = 1}^{#2} {#3}}

\newdimen\zzinstsymLTwidth
\newdimen\zzinstsymEQwidth
\newdimen\zzinstsymDiff
\newcommand{\instsymLeq}{%
    \settowidth{\zzinstsymLTwidth}{\text{\normalfont\tt<}}%
    \settowidth{\zzinstsymEQwidth}{\text{\normalfont=}}%
    \setlength{\zzinstsymDiff}{\zzinstsymEQwidth}%
    \addtolength{\zzinstsymDiff}{-\zzinstsymLTwidth}%
    \text{\raisebox{-0.22ex}{\normalfont=}%
%
    \hspace{-\zzinstsymEQwidth}%
    \hspace{0.5\zzinstsymDiff}%
    \raisebox{0.77ex}{\normalfont\tt<}}}
\newcommand{\instsymColon}{%
     \raisebox{-0.09ex}{\text{\normalfont{:}}}}
%
%
%
\newcommand{\instsyml}{\subtypingycolor{\instsymColon\hspace{0.05ex}\instsymLeq}}
\newcommand{\instsymr}{\subtypingycolor{\instsymLeq\hspace{0.05ex}\instsymColon}}
\newcommand{\instsymlop}{\mathrel{\instsyml}}
\newcommand{\instsymrop}{\mathrel{\instsymr}}

\newcommand{\instjudg}[4]{\ensuremath{{#1} \entails {#2} \instsymlop {#3} \ctxout{#4}}}
\newcommand{\instjudgr}[4]{\ensuremath{{#1} \entails {#3} \instsymrop {#2} \ctxout{#4}}}

\newcommand{\declsubjudgPf}[4] {\Pf{#1}{\entails}{{#2} \declsubtype {#3}}{#4}}
\newcommand{\subjudgPf}[5] {\Pf{#1}{\entails}{{#2} \subtype {#3} \ctxout{#4}}{#5}}
\newcommand{\substextendPf}[3] {\Pfmrg{{#1} \extendssym\,}{#2}{#3}}
\newcommand{\instjudgPf}[5]{\Pf{#1}{\entails}{{#2} {\;\instsyml\;} {#3} \ctxout{#4}}{#5}}
\newcommand{\instjudgrPf}[5]{\Pf{#1}{\entails}{{#3} {\;\instsymr\;} {#2} \ctxout{#4}}{#5}}

\newcommand{\chkcolor}{dBlue}
\newcommand{\syncolor}{dRed}
\newcommand{\appcolor}{dDkGreen}
\newcommand{\chk}{\mathrel{\mathcolor{\chkcolor}{\Leftarrow}}}
\newcommand{\uncoloredsyn}{{\Rightarrow}}
\newcommand{\syn}{\mathrel{\mathcolor{\syncolor}{\uncoloredsyn}}}

\newcommand{\appsep}{\;{\mathcolor{\appcolor}{\bullet}}\;}
%
\newcommand{\app}{\mathrel{\mathcolor{\appcolor}{{\uncoloredsyn}\hspace{-1.2ex}{\uncoloredsyn}}}}

% unification
\newcommand{\unicolor}{dAlsoPurple}
\newcommand{\unisep}{\;{\mathcolor{\unicolor}{\cap}}\;}
\newcommand{\uni}{\mathrel{\mathcolor{\unicolor}{{\uncoloredsyn}\hspace{-1.2ex}{\uncoloredsyn}}}}


\newcommand{\unijudg}[5]{\ensuremath{{#1} \entails {#2} \unisep {#3} \uni {#4} \ctxout{#5}}}

\newcommand{\chkjudg}[4]{\ensuremath{{#1} \entails {#2} \chk {#3} \ctxout{#4}}}
\newcommand{\appjudg}[5]{\ensuremath{{#1} \entails {#3} \appsep {#2} \app {#4} \ctxout{#5}}}
\newcommand{\synjudg}[4]{\ensuremath{{#1} \entails {#2} \syn {#3} \ctxout{#4}}}

\DeclareMathOperator*{\chks}{\vphantom{\lim}\!\mathrel{\!\mathcolor{\chkcolor}{\Leftarrow}}}
\newcommand{\chkjudgs}[5]{\ensuremath{{#1} \entails {#2} \chks_{i = 1}^{#3} {#4} \ctxout{#5}}}

\DeclareMathOperator*{\syns}{\vphantom{\lim}\!\mathrel{\!\mathcolor{\syncolor}{\uncoloredsyn}}}
\newcommand{\synjudgs}[5]{\ensuremath{{#1} \entails {#2} \syns_{i = 1}^{#3} {#4} \ctxout{#5}}}

\DeclareMathOperator*{\unis}{\mathrel{\mathcolor{\unicolor}{{\uncoloredsyn}\hspace{-1.2ex}{\uncoloredsyn}}}}
\newcommand{\unijudgs}[6]{\ensuremath{{#1} \entails #2 \unisep #3 \unis_{i = 1}^{#4} {#5} \ctxout{#6}}}

\newcommand{\declchkjudg}[3]{\ensuremath{{#1} \entails {#2} \chk {#3}}}
\newcommand{\declappjudg}[4]{\ensuremath{#1} \entails {#3} \appsep {#2}  \app {#4}}

\newcommand{\family}[2]{\{ {#1}_i \}_1^{#2}}
